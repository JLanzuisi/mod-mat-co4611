% % % % % % % % % % % % % % % % % % % % % % % % % % % % % %
% This TeX file is auto-generated from an rmarkdown source.
% % % % % % % % % % % % % % % % % % % % % % % % % % % % % %
%
% Options for packages loaded elsewhere
\PassOptionsToPackage{unicode}{hyperref}
\PassOptionsToPackage{hyphens}{url}
%
\documentclass[
  10pt,
  spanish,
  twocolumn,DIV=18,toc=flat]{scrartcl}
\usepackage{lmodern}
\usepackage{amssymb,amsmath}
\usepackage{ifxetex,ifluatex}
\ifnum 0\ifxetex 1\fi\ifluatex 1\fi=0 % if pdftex
  \usepackage[T1]{fontenc}
  \usepackage[utf8]{inputenc}
  \usepackage{textcomp} % provide euro and other symbols
\else % if luatex or xetex
  \usepackage{unicode-math}
  \defaultfontfeatures{Scale=MatchLowercase}
  \defaultfontfeatures[\rmfamily]{Ligatures=TeX,Scale=1}
\fi
% Use upquote if available, for straight quotes in verbatim environments
\IfFileExists{upquote.sty}{\usepackage{upquote}}{}
\IfFileExists{microtype.sty}{% use microtype if available
  \usepackage[]{microtype}
  \UseMicrotypeSet[protrusion]{basicmath} % disable protrusion for tt fonts
}{}
\makeatletter
\@ifundefined{KOMAClassName}{% if non-KOMA class
  \IfFileExists{parskip.sty}{%
    \usepackage{parskip}
  }{% else
    \setlength{\parindent}{0pt}
    \setlength{\parskip}{6pt plus 2pt minus 1pt}}
}{% if KOMA class
  \KOMAoptions{parskip=half}}
\makeatother
\usepackage{xcolor}
\IfFileExists{xurl.sty}{\usepackage{xurl}}{} % add URL line breaks if available
\IfFileExists{bookmark.sty}{\usepackage{bookmark}}{\usepackage{hyperref}}
\hypersetup{
  pdftitle={Tarea 2},
  pdfauthor={Jhonny Lanzuisi},
  pdflang={es-ES},
  hidelinks,
  pdfcreator={LaTeX via pandoc}}
\urlstyle{same} % disable monospaced font for URLs
\usepackage{color}
\usepackage{fancyvrb}
\newcommand{\VerbBar}{|}
\newcommand{\VERB}{\Verb[commandchars=\\\{\}]}
\DefineVerbatimEnvironment{Highlighting}{Verbatim}{commandchars=\\\{\}}
% Add ',fontsize=\small' for more characters per line
\newenvironment{Shaded}{}{}
\newcommand{\AlertTok}[1]{\textcolor[rgb]{1.00,0.00,0.00}{\textbf{#1}}}
\newcommand{\AnnotationTok}[1]{\textcolor[rgb]{0.38,0.63,0.69}{\textbf{\textit{#1}}}}
\newcommand{\AttributeTok}[1]{\textcolor[rgb]{0.49,0.56,0.16}{#1}}
\newcommand{\BaseNTok}[1]{\textcolor[rgb]{0.25,0.63,0.44}{#1}}
\newcommand{\BuiltInTok}[1]{#1}
\newcommand{\CharTok}[1]{\textcolor[rgb]{0.25,0.44,0.63}{#1}}
\newcommand{\CommentTok}[1]{\textcolor[rgb]{0.38,0.63,0.69}{\textit{#1}}}
\newcommand{\CommentVarTok}[1]{\textcolor[rgb]{0.38,0.63,0.69}{\textbf{\textit{#1}}}}
\newcommand{\ConstantTok}[1]{\textcolor[rgb]{0.53,0.00,0.00}{#1}}
\newcommand{\ControlFlowTok}[1]{\textcolor[rgb]{0.00,0.44,0.13}{\textbf{#1}}}
\newcommand{\DataTypeTok}[1]{\textcolor[rgb]{0.56,0.13,0.00}{#1}}
\newcommand{\DecValTok}[1]{\textcolor[rgb]{0.25,0.63,0.44}{#1}}
\newcommand{\DocumentationTok}[1]{\textcolor[rgb]{0.73,0.13,0.13}{\textit{#1}}}
\newcommand{\ErrorTok}[1]{\textcolor[rgb]{1.00,0.00,0.00}{\textbf{#1}}}
\newcommand{\ExtensionTok}[1]{#1}
\newcommand{\FloatTok}[1]{\textcolor[rgb]{0.25,0.63,0.44}{#1}}
\newcommand{\FunctionTok}[1]{\textcolor[rgb]{0.02,0.16,0.49}{#1}}
\newcommand{\ImportTok}[1]{#1}
\newcommand{\InformationTok}[1]{\textcolor[rgb]{0.38,0.63,0.69}{\textbf{\textit{#1}}}}
\newcommand{\KeywordTok}[1]{\textcolor[rgb]{0.00,0.44,0.13}{\textbf{#1}}}
\newcommand{\NormalTok}[1]{#1}
\newcommand{\OperatorTok}[1]{\textcolor[rgb]{0.40,0.40,0.40}{#1}}
\newcommand{\OtherTok}[1]{\textcolor[rgb]{0.00,0.44,0.13}{#1}}
\newcommand{\PreprocessorTok}[1]{\textcolor[rgb]{0.74,0.48,0.00}{#1}}
\newcommand{\RegionMarkerTok}[1]{#1}
\newcommand{\SpecialCharTok}[1]{\textcolor[rgb]{0.25,0.44,0.63}{#1}}
\newcommand{\SpecialStringTok}[1]{\textcolor[rgb]{0.73,0.40,0.53}{#1}}
\newcommand{\StringTok}[1]{\textcolor[rgb]{0.25,0.44,0.63}{#1}}
\newcommand{\VariableTok}[1]{\textcolor[rgb]{0.10,0.09,0.49}{#1}}
\newcommand{\VerbatimStringTok}[1]{\textcolor[rgb]{0.25,0.44,0.63}{#1}}
\newcommand{\WarningTok}[1]{\textcolor[rgb]{0.38,0.63,0.69}{\textbf{\textit{#1}}}}
\usepackage{graphicx}
\makeatletter
\def\maxwidth{\ifdim\Gin@nat@width>\linewidth\linewidth\else\Gin@nat@width\fi}
\def\maxheight{\ifdim\Gin@nat@height>\textheight\textheight\else\Gin@nat@height\fi}
\makeatother
% Scale images if necessary, so that they will not overflow the page
% margins by default, and it is still possible to overwrite the defaults
% using explicit options in \includegraphics[width, height, ...]{}
\setkeys{Gin}{width=\maxwidth,height=\maxheight,keepaspectratio}
% Set default figure placement to htbp
\makeatletter
\def\fps@figure{htbp}
\makeatother
\setlength{\emergencystretch}{3em} % prevent overfull lines
\providecommand{\tightlist}{%
  \setlength{\itemsep}{0pt}\setlength{\parskip}{0pt}}
\setcounter{secnumdepth}{5}
%\RecustomVerbatimEnvironment{Highlighting}{Verbatim}{commandchars=\\\{\},fontfamily=mlmr,frame=leftline,numbers=left,numbersep=2.5pt}

\setlength{\fboxsep}{5pt}
\setlength{\columnsep}{20pt}

\setkomafont{title}{\normalfont\sffamily}
\setkomafont{disposition}{\normalfont\sffamily}
\setkomafont{subtitle}{\normalfont\large\sffamily}
\setkomafont{section}{\normalfont\Large\sffamily}
\setkomafont{subsection}{\normalfont\large\sffamily}

\titlehead{Universidad Simón Bolívar\hfill Matemáticas Puras y Aplicadas}
\usepackage{mlmodern}
\usepackage[scale=.8]{plex-mono}
\ifxetex
  % Load polyglossia as late as possible: uses bidi with RTL langages (e.g. Hebrew, Arabic)
  \usepackage{polyglossia}
  \setmainlanguage[]{spanish}
\else
  \usepackage[shorthands=off,main=spanish]{babel}
\fi

\title{Tarea 2}
\usepackage{etoolbox}
\makeatletter
\providecommand{\subtitle}[1]{% add subtitle to \maketitle
  \apptocmd{\@title}{\par {\large #1 \par}}{}{}
}
\makeatother
\subtitle{Modelaje Matemático}
\author{Jhonny Lanzuisi}
\date{16 de Junio de 2022}

\begin{document}
\maketitle

{
\setcounter{tocdepth}{3}
\tableofcontents
}
\begin{Shaded}
\begin{Highlighting}[]
\ImportTok{import}\NormalTok{ sympy }\ImportTok{as}\NormalTok{ sym}
\ImportTok{import}\NormalTok{ numpy }\ImportTok{as}\NormalTok{ np}
\ImportTok{from}\NormalTok{ scipy.integrate }\ImportTok{import}\NormalTok{ solve\_ivp}
\ImportTok{import}\NormalTok{ matplotlib.style}
\ImportTok{import}\NormalTok{ matplotlib.pyplot }\ImportTok{as}\NormalTok{ plt}
\ImportTok{import}\NormalTok{ matplotlib}

\NormalTok{matplotlib.style.use(}\StringTok{\textquotesingle{}seaborn\textquotesingle{}}\NormalTok{)}

\NormalTok{matplotlib.rcParams.update(\{}
    \StringTok{\textquotesingle{}figure.autolayout\textquotesingle{}}\NormalTok{: }\VariableTok{True}\NormalTok{,}
\NormalTok{\})}

\KeywordTok{def}\NormalTok{ lp(input\_string):}
   \BuiltInTok{print}\NormalTok{(}\StringTok{\textquotesingle{}$$\textquotesingle{}} \OperatorTok{+}\NormalTok{ sym.latex(input\_string) }\OperatorTok{+} \StringTok{\textquotesingle{}$$\textquotesingle{}}\NormalTok{)}
\end{Highlighting}
\end{Shaded}

\hypertarget{primera-pregunta}{%
\section{Primera pregunta}\label{primera-pregunta}}

Sabemos que podemos obtener \(t\) de la relación: \[
    \frac{R(t)}{R(0)} = e^{\lambda t}.
\]

Debemos primero determinar \(\lambda\). Supongamos que la cantidad
inicial de carbono es \(1\), que representa el 100\%, entonces usando el
hecho de que la vida media es \(5568\) tenemos \[
    N(5568) = \frac{1}{2} = 1 e^{-\lambda 5568}.
\] y despejando se obtiene \[
    -\frac{\log(1/2)}{5568} = \lambda.
\]

\begin{Shaded}
\begin{Highlighting}[]
\ImportTok{import}\NormalTok{ math}

\NormalTok{lmda }\OperatorTok{=}\NormalTok{ math.log(}\DecValTok{1}\OperatorTok{/}\DecValTok{2}\NormalTok{)}\OperatorTok{/}\DecValTok{5568}
\BuiltInTok{print}\NormalTok{(}\StringTok{"lambda:"}\NormalTok{, lmda)}
\end{Highlighting}
\end{Shaded}

\begin{verbatim}
## lambda: -0.0001244876401867718
\end{verbatim}

Tenemos entonces que: \[
    \frac{\log(R(t)/R(0))}{\lambda} = t
\] y al sustituir los valores en una calculadora podemos hallar el valor
de \(t\):

\begin{Shaded}
\begin{Highlighting}[]
\BuiltInTok{print}\NormalTok{(}\StringTok{"t:"}\NormalTok{, math.log(}\FloatTok{0.97}\OperatorTok{/}\FloatTok{6.68}\NormalTok{)}\OperatorTok{/}\NormalTok{lmda)}
\end{Highlighting}
\end{Shaded}

\begin{verbatim}
## t: 15500.150795197595
\end{verbatim}

\hypertarget{segunda-pregunta}{%
\section{Segunda pregunta}\label{segunda-pregunta}}

\hypertarget{a}{%
\subsection{(a)}\label{a}}

Los puntos de equilibrio son aquellos en que la derivada se anula, en
este caso \(x=0\) implica que \(e^{-x^2} = 1\) y \(\dot{x} = 0\).

Para estudiar la estabilidad queremos ver como se comporta la derivada
al variar \(x\) ligeramente. Si \(x\) es pequeño y positivo entonces la
derivada es positiva pues \(e^{-x^2} < 1\). Si, por otro lado, \(x\) es
pequeño y negativo entonces \(e^{-x^2} > 1\) y la derivada es negativa.

En el diagrama de fase luce entonces como la figura \ref{p1}.

\begin{Shaded}
\begin{Highlighting}[]
\NormalTok{x }\OperatorTok{=}\NormalTok{ sym.symbols(}\StringTok{\textquotesingle{}x\textquotesingle{}}\NormalTok{)}

\NormalTok{p1 }\OperatorTok{=}\NormalTok{ sym.plot(sym.exp(x), (x,}\OperatorTok{{-}}\DecValTok{1}\NormalTok{,}\DecValTok{1}\NormalTok{), show}\OperatorTok{=}\VariableTok{False}\NormalTok{)}
\end{Highlighting}
\end{Shaded}

\hypertarget{b}{%
\subsection{(b)}\label{b}}

La derivada se hace cero cuando \[
    x(a-x^2)
\] se hace cero. Lo que ocurre en \(x=0\) y \(x=\pm\sqrt{a}\).

Si \(a=0\) entonces el único punto de equilibrio es el cero. Si hacemos
\(x\) pequeño y positivo entonces la derivada es negativa y cuando \(x\)
es pequeño y negativo la derivada es positiva. El diagrama de fase es
entonces como en la figura \ref{p2}, y el punto \(x=0\) es estable.

\begin{Shaded}
\begin{Highlighting}[]
\NormalTok{p2 }\OperatorTok{=}\NormalTok{ sym.plot(}\OperatorTok{{-}}\NormalTok{x}\OperatorTok{**}\DecValTok{3}\NormalTok{, (x,}\OperatorTok{{-}}\DecValTok{1}\NormalTok{,}\DecValTok{1}\NormalTok{), show}\OperatorTok{=}\VariableTok{False}\NormalTok{)}
\end{Highlighting}
\end{Shaded}

Si \(a<0\) la raíz de \(a\) no es un número real y el único punto de
equilibrio es nuevamente el cero, con el mismo diagrama de la figura
\ref{p2}.

Si \(a>0\) los puntos de equilibrio son el cero y \(\pm\sqrt{a}\). Para
el punto \(\sqrt{a}\) tenemos que para \(x^*>\sqrt{a}\) la derivada es
negativa y para \(x^*<\sqrt{a}\) la derivada es positiva. Para el punto
\(-\sqrt{a}\) tenemos que para \(x^*>-\sqrt{a}\) la derivada es negativa
y cuando \(x^*<-\sqrt{a}\) la derivada es positiva. El diagrama de fase
esta en la figura \ref{p3}. De donde se tiene que los puntos
\(\pm\sqrt{a}\) son estables y \(0\) es inestable.

\begin{Shaded}
\begin{Highlighting}[]
\NormalTok{p3 }\OperatorTok{=}\NormalTok{ sym.plot(show}\OperatorTok{=}\VariableTok{False}\NormalTok{, legend}\OperatorTok{=}\VariableTok{True}\NormalTok{)}
\ControlFlowTok{for}\NormalTok{ i }\KeywordTok{in} \BuiltInTok{range}\NormalTok{(}\DecValTok{0}\NormalTok{,}\DecValTok{5}\NormalTok{):}
    \ControlFlowTok{if}\NormalTok{ i }\OperatorTok{!=} \DecValTok{0}\NormalTok{:}
\NormalTok{        p }\OperatorTok{=}\NormalTok{ sym.plot(}
\NormalTok{            x}\OperatorTok{*}\NormalTok{(i}\OperatorTok{{-}}\NormalTok{x}\OperatorTok{**}\DecValTok{2}\NormalTok{),}
\NormalTok{            (x,}\OperatorTok{{-}}\DecValTok{2}\NormalTok{,}\DecValTok{2}\NormalTok{),}
\NormalTok{            show}\OperatorTok{=}\VariableTok{False}\NormalTok{,}
\NormalTok{            label}\OperatorTok{=}\StringTok{\textquotesingle{}a= \textquotesingle{}} \OperatorTok{+} \BuiltInTok{str}\NormalTok{(i)}
\NormalTok{        )}
\NormalTok{        p3.append(p[}\DecValTok{0}\NormalTok{])}
\end{Highlighting}
\end{Shaded}

\onecolumn
\appendix

\hypertarget{gruxe1ficos}{%
\section{Gráficos}\label{gruxe1ficos}}

\begin{figure}
\includegraphics[width=1\linewidth]{p1} \caption{\label{p1} Diagrama de fase: $\dot{x} = 1-e^{-x^2}$}\label{fig:unnamed-chunk-7}
\end{figure}

\begin{figure}
\includegraphics[width=1\linewidth]{p2} \caption{\label{p2} Diagrama de fase: $\dot{x} = -x^3$}\label{fig:unnamed-chunk-8}
\end{figure}

\begin{figure}
\includegraphics[width=1\linewidth]{p3} \caption{\label{p3} Diagrama de fase: $\dot{x} = x(a-x^2)$}\label{fig:unnamed-chunk-9}
\end{figure}

\end{document}
